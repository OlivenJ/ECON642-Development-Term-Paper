\documentclass[12pt]{article}
\usepackage{amsmath}
\usepackage{mathptmx}
\usepackage{setspace}
\usepackage{amssymb}
\usepackage{float}
\usepackage{graphicx}
\usepackage{booktabs}
\usepackage{subfigure}
\usepackage[margin=2.5cm]{geometry}
\usepackage[title]{appendix}
\usepackage{bm}
\usepackage{tcolorbox}
\usepackage{apacite}
\usepackage{lipsum}  
\onehalfspacing

\usepackage{diagbox}



\usepackage{fontspec}
\setmainfont{Times New Roman}
\addtolength{\jot}{0.5em}
\linespread{1}

\begin{document}
\begin{titlepage}
\begin{center}
\vspace*{1cm}
\Huge
\textbf{Econ642 Development Economic}

\vspace{0.5cm}
\LARGE
Women Participation and National Health Expenditure
\\
\Large
Research Proposal

\vspace{1.5 cm}
\textbf{Zhiyuan Jiang\\}
\vfill

\vspace{0.8cm}
 
\Large
\today
\end{center}
\end{titlepage}
\tableofcontents
\thispagestyle{empty}
\newpage

\begin{quotation}
	Women belong in all places where decisions are being made
\begin{flushright}
-Ruth Bader Ginsburg
\end{flushright}
\end{quotation}

\section{Introduction and Motivation}
The central topic of this research is the relationship between women's participation in the law-making process and the well-being of the country's citizens.
More specifically, this research is trying to discover the relationship between the ratio of female lawmakers in a country's legislative body and the national health care expenditure.

In 2015, United Nations (UN) presented a new framework aiming to improve the sustainable development of the world, naming it as "Sustainable Development Goals" (SDGs).
This framework includes 17 bulletins, and the fifth goal is "achieve gender equality and empower all women and girls".
This high priority of gender equality indicates the importance of this issue with regard to the general development of human welfare.

In fact, the relationship between economic development and the empowerment of women had been long discussed by various scholars.
\citeA{Duflo2012} documented a series of papers researching the relationship between women's empowerment and economic development.
In general, the research agrees on a positive relationship between women's empowerment and economic development, and the relationship goes in both ways.
In the paper, \citeauthor{Duflo2012} admitted that although with limitations, women's empowerment can lead to improvements such as children's health and nutrition.
One potential mechanism that enables such improvement is budget allocation.
Women can influence the item being purchased, and therefore influence the welfare.
This conclusion, from the micro-level, had been found by numerous researchers.
\citeA{Hoddinott1995} studied the data from Cote d'Ivoire and found that women tend to spend household income on family-friendly items such as nutrition rather than alcohol or cigarette.
Spending on nutrition has a positive influence on children's health.
Similarly, \citeA{Quisumbing2003} argues that in some countries the increase of female members' assets will lead to the increase of expenditure on children's education. 
The evidence above arguable proved that at least at the household level, females tend to be better budget makers than males when on the issues of health and education. 

Not only at the household level, but the presentation of women in other fields has also been proved to have a positive influence.
In governing, \citeA{Dollar2001} provided evidence to show that the involvement of female officers in government can reduce the overall corruption level.
But this result had been questioned by other scholars (such as \citeNP{Sung2003}) since this effect can be caused by a better democratic system that encourages more women participation and dampens the corruption, rather than the other way around. 
Taking this measurement into account, however, later research still finds the benefits of the involvement of females.

Especially in the legislative body where lawmakers have the power of making laws and designing budgets.
\citeA{Jayasuriya2013} collected data from over 100 countries and concludes that the country with a higher participation rate of women in the law-making process tends to have a higher economic growth rate in general.
As a subjective indicator, \citeA{York2014} presented the result that people tend to have a higher life-satisfaction rate if their national parliament or house of deputies has a higher ratio of female members.  
The influence of female lawmakers also extends to other more specific fields.
For example, through the budget control, \citeA{Salahodjaev2020} showed an "S" shape relationship between a country's deforestation level and proportion of women members in the legislative body. 

With all the research presented above, the relationship between women's participation in the parliament and the national health expenditure, which is closely related to the well-being of citizens, especially after the covid-19 pandemic, has rarely been discussed. 
Most of the studies with regard the health expenditures are focused on either the low-level government officers (for instance the municipal level in \citeNP{Funk2018}) or the administrative branch of the national government (see \citeNP{Mavisakalyan2014}).
This research tries to make up the gap by discovering the relationship between the ratio of female lawmakers and the national health expenditure through a data-driven quantitative centric method.
The methodology and data set will be discussed in the following sections.


\section{Methodology Data and Preliminary Analsysis}
This research will employ a simple multi-variable Ordinary Least Square (OLS) model.
The left-hand side variable of the model will be the ratio of the country's health care expenditure to its total GDP.
The primary right-hand side controlled variable will be the ratio of female members of the country's legislative body.
Suggested by the common practice in the field of development economic and health economic, other controlled variables that relates to the health care budget size that will be used in this research model include:
\begin{itemize}
	\item Ratio of population aged 64 and above to total population
	\item GDP per capita
	\item Variables measures the prevalence of diseases such as AIDS or TB
	\item Democracy level
	\item Received foreign aid Amount in US dollar per capita
	\item Labor participation rate of female

\end{itemize}
The first three items from the above list are common variables when measuring a country's health-related expenditure
The foreign aid variable will strongly correlate with the developing countries, especially countries in Africa that received a large amount of Official development assistance (ODA) to help build the health system.
The democracy index variable will test the different influences of legislative bodies in democratic countries and non-democratic countries.
The last labor participation rate of females helps solve the problem presented by \citeA{Sung2003} about the arguments between better gender and system. 
Other related variables such as the girls' enrollment rate in the secondary education system will also be used in the research as a substitute to the participation rate to test the robustness of the model. 


The primary source of the data is the World Bank Open Data\footnote{visit: https://data.worldbank.org/}.
And the measurement of democracy level comes from the Freedom House's freedom score\footnote{https://freedomhouse.org/countries/freedom-world/scores}.


\section{Expectation and Extension}


The foreign aid variable can be simplified as a dummy variable to measure the effect of ODA.
\newpage
\bibliographystyle{apacite}
\bibliography{export.bib}

\end{document}